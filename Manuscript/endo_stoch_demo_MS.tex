\documentclass[9pt,twocolumn,twoside,lineno]{pnas-new}
% Use the lineno option to display guide line numbers if required.

\newcommand{\tom}[2]{{\color{red}{#1}}\footnote{\textit{\color{red}{#2}}}}

\templatetype{pnasresearcharticle} % Choose template 
% {pnasresearcharticle} = Template for a two-column research article
% {pnasmathematics} %= Template for a one-column mathematics article
% {pnasinvited} %= Template for a PNAS invited submission
\title{Context-dependent host-microbe interactions in stochastic environments}

% Use letters for affiliations, numbers to show equal authorship (if applicable) and to indicate the corresponding author
\author[a,1]{Joshua C. Fowler}
\author[b]{Shaun Ziegler}
\author[b]{Kenneth D. Whitney} 
\author[b]{Jennifer A. Rudgers}
\author[a]{Tom E. X. Miller}


\affil[a]{Rice University, Department of BioSciences, Houston, TX, 77005}
\affil[b]{University of New Mexico, Department of Biology, Albuquerque, NM, 87131}

% Please give the surname of the lead author for the running footer
\leadauthor{Fowler} 

% Please add here a significance statement to explain the relevance of your work
\significancestatement{Authors must submit a 120-word maximum statement about the significance of their research paper written at a level understandable to an undergraduate educated scientist outside their field of speciality. The primary goal of the Significance Statement is to explain the relevance of the work in broad context to a broad readership. The Significance Statement appears in the paper itself and is required for all research papers.}

% Please include corresponding author, author contribution and author declaration information
\authorcontributions{Please provide details of author contributions here.}
\authordeclaration{Please declare any conflict of interest here.}
\correspondingauthor{\textsuperscript{1}To whom correspondence should be addressed. E-mail: jcf3\@rice.edu}

% Keywords are not mandatory, but authors are strongly encouraged to provide them. If provided, please include two to five keywords, separated by the pipe symbol, e.g:
\keywords{Keyword 1 $|$ Keyword 2 $|$ Keyword 3 $|$ ...} 

\begin{abstract}
Please provide an abstract of no more than 250 words in a single paragraph. Abstracts should explain to the general reader the major contributions of the article. References in the abstract must be cited in full within the abstract itself and cited in the text.
\end{abstract}

\dates{This manuscript was compiled on \today}
\doi{\url{www.pnas.org/cgi/doi/10.1073/pnas.XXXXXXXXXX}}

\usepackage{Sweave}
\begin{document}
\Sconcordance{concordance:endo_stoch_demo_MS.tex:endo_stoch_demo_MS.Rnw:%
1 45 1 1 0 293 1}


\maketitle
\thispagestyle{firststyle}
\ifthenelse{\boolean{shortarticle}}{\ifthenelse{\boolean{singlecolumn}}{\abscontentformatted}{\abscontent}}{}

% If your first paragraph (i.e. with the \dropcap) contains a list environment (quote, quotation, theorem, definition, enumerate, itemize...), the line after the list may have some extra indentation. If this is the case, add \parshape=0 to the end of the list environment.
%Introduction
\dropcap{A}long with increases in average temperatures, global climate change is driving increases in the variability of precipitation events, temperature extremes, and droughts \cite{IPCC2012managing, seneviratne2012changes, stocker2013technical}. Thus discerning the effects of variability on population dynamics and species interactions is pivotal to forecasting the future of ecological systems. Classic ecological theory predicts that long-term population growth rates will be reduced by environmental variability \cite{lewontin_population_1969,tuljapurkar_population_1982}. This stochastic variability means that populations can expect to experience both good and bad years. The long-term stochastic growth rate ($\lambda_s$), which is the long-run geometric mean of annual growth rates, captures this varability; the geometric mean will always be less than expected from the mean growth rate alone. 

Following Lewontin and Cohen \citep{lewontin_population_1969}, $\lambda_s$ can be approximated as: 
\begin{align}
 log(\lambda_s)  \approx log(\overline{\lambda}) - \frac{\sigma^2}{2\overline{\lambda}^2}
\end{align}

Where $\overline{\lambda}$ is the mean of annual population growth rates ($\lambda_t$) and $\sigma^2$ is the variance \citep{lewontin_population_1969}. Populations will increase over time if $\lambda_s$ is greater than 1, and can be expected to decrease if $\lambda_s$ is less than 1. Here, there are two pathways to increase $\lambda_s$:  (1) increasing the mean growth rate, and/or (2) reducing the variance in growth rates. That both mean and variance can determine fitness underlies understanding of which aspects of a species' life history influence its success \cite{pfister1998patterns} and has important implications for population viability analysis \cite{menges1990population}. The degree to which species interactions contribute to effects on the variance in growth rates is an underexplored question \cite{hilde_demographic_2020}.

Microbial symbioses are ubiquitous in nature and are crucial determinants of host fitness \cite{rodriguez2009fungal, mcfall2013animals} but their potential influence on responses to environmental variability is under-appreciated \cite{rudgers2020climate}. Across a broad range of taxa, mutualistic host-associated microbes provide protection from environmental stresses including drought, extreme temperatures, and enemies \cite{russell2006costs, brownlie2009symbiont, kivlin2013fungal,corbin2017heritable, hoadley2019host}. The role they play may be under-appreciated, and it can be difficult to quantify the net outcome of a given interaction because they are often viewed as being context-dependent where the magnitude of benefit depends on environmental conditions \cite{chamberlain2014context}. Rather than considering context-dependence as some unexplainable intricacy of species interactions, environmental variation opens up the possibility for interaction strength to vary through time \cite{jordano1994spatial, billick2003relative} and to influence the variance of population growth rates. 

Symbionts may provide benefits under harsh conditions when they are needed by their hosts, but be neutral or even costly under benign conditions (cite). Over time, this may lead symbiont-associated organisms to experience a reduction in variation in vital rates by reducing the frequency of extreme years (conceptual figure). Embracing context-dependence in this way, we reveal a novel mechanism by which symbionts can act as mutualists that may come to be of increasing importance in a more variable future.

Using long-term demographic data for grass species hosting Epichlo\"{e} fungal endophytes, we test the hypothesis that context-dependent benefits of microbial symbionts buffer hosts from the fitness consequences of environmental variability. These endophytes are common among cool-season grasses and are primarily vertically transmitted from parent to seed \cite{cheplick2009ecology}. While they have been associated with contributing to drought tolerance for their hosts, these benefits are commonly context-dependent\cite{cheplick2004recovery, kannadan2008endophyte, decunta2021systematic}. And so, we ask first how fungal endophytes influence the mean and interannual variance of their hosts' vital rates; next, we ask if these vital rate effects buffer variance in fitness and, if so, what is the relative importance of variance buffering vs. mean effects in the overall fitness impact of the symbiosis. To answer these questions, we use data from experimental plots originally established in 2007 to build structured, stochastic population models for seven species of grass hosts (\textit{Agrostis perennans}, \textit{Elymus villosus}, \textit{Elymus virginicus}, \textit{Festuca subverticillata},\textit{Lolium arundinaceum}, \textit{Poa alsodes} and \textit{Poa sylvestris}). These long-term plots which are censused annually contain either naturally symbiotic plants or those which have had their symbionts experimentally removed. Across 14 years, the data contain 31,216 individual transition years. Each annual census is a sample of weather variation, allowing us to construct climate-explicit population models, which we use to evaluate the imporotance ofo buffering under forecasted changes in the mean and variance of climate drivers.

\textcolor{red}{This paragraph is mostly talking off my head about results, but my idea is to include a brief statement of our results.\tom{}{Agree we will want a punchy summary that leaves readers wanting to continue into the Results.}}
Across species, we find that variance buffering by endophytes contributes (percentage) to population growth rates. While the effect is generally weaker than effects on the mean, we found that buffering was common in the most sensitive vital rates, and was most important for xxx species with xxx life history.


\section*{Results}

\section*{Discussion}

\section*{Guide to using this template on Overleaf}

Please note that whilst this template provides a preview of the typeset manuscript for submission, to help in this preparation, it will not necessarily be the final publication layout. For more detailed information please see the \href{http://www.pnas.org/site/authors/format.xhtml}{PNAS Information for Authors}.

If you have a question while using this template on Overleaf, please use the help menu (``?'') on the top bar to search for \href{https://www.overleaf.com/help}{help and tutorials}. You can also \href{https://www.overleaf.com/contact}{contact the Overleaf support team} at any time with specific questions about your manuscript or feedback on the template.

\subsection*{Author Affiliations}

Include department, institution, and complete address, with the ZIP/postal code, for each author. Use lower case letters to match authors with institutions, as shown in the example. Authors with an ORCID ID may supply this information at submission.

\subsection*{Submitting Manuscripts}

All authors must submit their articles at \href{http://www.pnascentral.org/cgi-bin/main.plex}{PNAScentral}. If you are using Overleaf to write your article, you can use the ``Submit to PNAS'' option in the top bar of the editor window. 

\subsection*{Format}

Many authors find it useful to organize their manuscripts with the following order of sections;  Title, Author Affiliation, Keywords, Abstract, Significance Statement, Results, Discussion, Materials and methods, Acknowledgments, and References. Other orders and headings are permitted.

\subsection*{Manuscript Length}

PNAS generally uses a two-column format averaging 67 characters, including spaces, per line. The maximum length of a Direct Submission research article is six pages and a Direct Submission Plus research article is ten pages including all text, spaces, and the number of characters displaced by figures, tables, and equations.  When submitting tables, figures, and/or equations in addition to text, keep the text for your manuscript under 39,000 characters (including spaces) for Direct Submissions and 72,000 characters (including spaces) for Direct Submission Plus.

\subsection*{References}

References should be cited in numerical order as they appear in text; this will be done automatically via bibtex, e.g. . All references should be included in the main manuscript file.  

\subsection*{Data Archival}

PNAS must be able to archive the data essential to a published article. Where such archiving is not possible, deposition of data in public databases, such as GenBank, ArrayExpress, Protein Data Bank, Unidata, and others outlined in the Information for Authors, is acceptable.

\subsection*{Language-Editing Services}
Prior to submission, authors who believe their manuscripts would benefit from professional editing are encouraged to use a language-editing service (see list at www.pnas.org/site/authors/language-editing.xhtml). PNAS does not take responsibility for or endorse these services, and their use has no bearing on acceptance of a manuscript for publication. 


\subsection*{Digital Figures}

Only TIFF, EPS, and high-resolution PDF for Mac or PC are allowed for figures that will appear in the main text, and images must be final size. Authors may submit U3D or PRC files for 3D images; these must be accompanied by 2D representations in TIFF, EPS, or high-resolution PDF format.  Color images must be in RGB (red, green, blue) mode. Include the font files for any text. 

Figures and Tables should be labelled and referenced in the standard way using the \verb|\label{}| and \verb|\ref{}| commands.

Figure \ref{} shows an example of how to insert a column-wide figure. To insert a figure wider than one column, please use the \verb|\begin{figure*}...\end{figure*}| environment. Figures wider than one column should be sized to 11.4 cm or 17.8 cm wide. Use \verb|\begin{SCfigure*}...\end{SCfigure*}| for a wide figure with side captions.

\subsection*{Tables}
In addition to including your tables within this manuscript file, PNAS requires that each table be uploaded to the submission separately as a Table file.  Please ensure that each table .tex file contains a preamble, the \verb|\begin{document}| command, and the \verb|\end{document}| command. This is necessary so that the submission system can convert each file to PDF.

\subsection*{Single column equations}

Authors may use 1- or 2-column equations in their article, according to their preference.

To allow an equation to span both columns, use the \verb|\begin{figure*}...\end{figure*}| environment mentioned above for figures.

Note that the use of the \verb|widetext| environment for equations is not recommended, and should not be used. 

\begin{figure*}[bt!]
\begin{align*}
(x+y)^3&=(x+y)(x+y)^2\\
       &=(x+y)(x^2+2xy+y^2) \numberthis \label{eqn:example} \\
       &=x^3+3x^2y+3xy^3+x^3. 
\end{align*}
\end{figure*}


\begin{table}%[tbhp]
\centering
\caption{Comparison of the fitted potential energy surfaces and ab initio benchmark electronic energy calculations}
\begin{tabular}{lrrr}
Species & CBS & CV & G3 \\
\midrule
1. Acetaldehyde & 0.0 & 0.0 & 0.0 \\
2. Vinyl alcohol & 9.1 & 9.6 & 13.5 \\
3. Hydroxyethylidene & 50.8 & 51.2 & 54.0\\
\bottomrule
\end{tabular}

\addtabletext{nomenclature for the TSs refers to the numbered species in the table.}
\end{table}

\subsection*{Supporting Information (SI)}

Authors should submit SI as a single separate PDF file, combining all text, figures, tables, movie legends, and SI references.  PNAS will publish SI uncomposed, as the authors have provided it.  Additional details can be found here: \href{http://www.pnas.org/page/authors/journal-policies}{policy on SI}.  For SI formatting instructions click \href{https://www.pnascentral.org/cgi-bin/main.plex?form_type=display_auth_si_instructions}{here}.  The PNAS Overleaf SI template can be found \href{https://www.overleaf.com/latex/templates/pnas-template-for-supplementary-information/wqfsfqwyjtsd}{here}.  Refer to the SI Appendix in the manuscript at an appropriate point in the text. Number supporting figures and tables starting with S1, S2, etc.

Authors who place detailed materials and methods in an SI Appendix must provide sufficient detail in the main text methods to enable a reader to follow the logic of the procedures and results and also must reference the SI methods. If a paper is fundamentally a study of a new method or technique, then the methods must be described completely in the main text.

\subsubsection*{SI Datasets} 

Supply Excel (.xls), RTF, or PDF files. This file type will be published in raw format and will not be edited or composed.


\subsubsection*{SI Movies}

Supply Audio Video Interleave (avi), Quicktime (mov), Windows Media (wmv), animated GIF (gif), or MPEG files and submit a brief legend for each movie in a Word or RTF file. All movies should be submitted at the desired reproduction size and length. Movies should be no more than 10 MB in size.


\subsubsection*{3D Figures}

Supply a composable U3D or PRC file so that it may be edited and composed. Authors may submit a PDF file but please note it will be published in raw format and will not be edited or composed.


\matmethods{
\subsection*{Natural history of grass-endophyte symbiosis}
To study the effects of context-dependent microbial symbiosis, we focused on \emph{Epichloë} fungal endophytes, which live in the aboveground tissue of many species of cool-season grasses and grow into their hosts' seeds where they can be transmitted vertically from mother to offspring plants. This vertical transmission couples host and symbiont fitness and leads to the expectation that the interaction be mutualistic, else the fungi cause their host to be selected out of the population \citep{fine1975vectors, douglas1998host, rudgers2009fungus}. While there are demonstrated benefits against herbivory\citep{brem2001epichloe} and under drought stress \citep{hamilton2012new} for some host species, these interactions outcomes are commonly context-dependent \citep{cheplick2004recovery, kannadan2008endophyte}.

\subsection*{Plant propagation and endophyte removal}
Seeds from naturally infected populations of seven species of cool-season grasses (\emph{Agrostis perennans}, \emph{Elymus villosus}, \emph{Elymus virginicus}, \emph{Festuca subverticillata}, \emph{Lolium arundinaceum}, \emph{Poa alsodes}, and \emph{Poa sylvestris}) were collected during the 2006 growing season from Lilly Dickie Woods (39.238533, -86.218150) and the Bayles Road Teaching and Research Preserve (39.220167, -86.542683) in Brown Co. IN. To reduce confounding genotype effects, seeds with shared maternal ancestry were disinfected with heat treatments \tom{(6d in a drying oven at 60$^{\circ}$ C for \emph{E. villosus}, \emph{E. virginicus}, \emph{F. subverticillata},  and \emph{L. arundinaceum}; 7d in a drying oven at 60$^{\circ}$ C for \emph{P. alsodes}, and \emph{P. sylvestris}; and 12 min. in a hot water bath at 60$^{\circ}$ C for \emph{A. perennans})}{need to double check methods for temp, duration, etc.} or left naturally infected. Seeds were surface sterilized with bleach and cold stratified for {\color{red}??? weeks}, then germinated in a growth chamber before being transferred to the greenhouse at Indiana University and allowed to grow for XXXX weeks. We confirmed the endophyte status of these plants using leaf peels, where tissue from the leaf sheath is stained with aniline blue dye and examined for the presence of fungal hyphae \citep{bacon2018stains}. Then, we established the experimental plots with \tom{vegetatively propogated clones of similar sizes from the plants}{not sure this happened} to reduce the potential for negative side effects of heat treatments \cite{rudgers2009benefits}.

\subsection*{Experimental design and data collection}
During the spring of 2007, we established 10 3x3 plots for \emph{A. perennans}, \emph{E. villosus}, \emph{E. virginicus}, \emph{F. subverticillata}, and \emph{L. arundinaceum}  and 18 plots for \emph{P. alsodes} and \emph{P. sylvestris}. For each species, an equal number of plots were randomly assigned to each endophyte status, E+ or E-. Each plot was planted with 20 evenly spaced symbiotic or symbiont-free individuals respectively and each plant marked with aluminum tags. 

Each summer starting in 2007, we censused all individuals in each plot for survival, growth and reproduction, garnering a dataset covering 14 years that contains 31,216 individual transition years. After clearing out leaf litter, for each plant alive in the previous year, we marked its survival and measured its size as a count of the number of tillers. Further, we collected reproductive data by counting the number of reproductive tillers, and then counting the number of seed-bearing spikelets on up to three of those reproductive tillers. In 2009, we took additional counts of seeds per inflorescence. Together, we use these measurements to estimate seed production. In each plot, we also survey for and tag any unmarked individuals. New recruits typically have one tiller and are non-reproductive, but we also find and tag any individuals that may have been missed in previous censuses.

We typically expect plots of each endophyte status to maintain their status as the fungus is almost entirely vertically transmitted, and plots are {\color{red}spaced at least 5 m apart}, limiting the possibility for unwanted dispersal between plots or horizontal transmission of the fungus. Seeds from reproductive individuals are opportunistically taken and scored for their endophyte status. Overall, these scores reflect a 97.5\%  faithfulness of recruits to their expected endophyte status across species and plots (Supplement data).

\subsection*{Demographic modeling}
Armed with this demographic data, we next constructed size-structured, stochastic population models. This model describes transitions between sizes (measured as a count of tillers) from one year to the next. For all species, we include a 1 year reproductive delay in the population model following the observation that these newly recruited plants are rarely observed flowering in their first year. Our population model can be expressed as:

\begin{equation}
\label{eq:matrixmodel}
\mathbf{n}_{t+1} = \mathbf{A}\mathbf{n}_{t}
\end{equation}

where $\mathbf{n}_{t+1}$ is a vector of abundances across sizes in year t+1 for each species and endophyte status.

\begin{equation}
\label{eq:nvector}
\mathbf{n}_{t+1} = \begin{bmatrix} size^{sdlg} \\ size_{i} \\ . \\ .\\ . \\ size_{N} \end{bmatrix} 
\end{equation}

and $\mathbf{A}$ is expressed as a N+1 x N+1 matrix:
\begin{equation}
\label{eq:Amatrix}
\mathbf{A} = \begin{bmatrix} 0 & F_{i}  & . & . & F_{N} \\
                            T^{sdlg} & T_{i}  & . & . & .\\
                            . & .  & . & . & .\\
                            . & .  & . & . & .\\
                            T^{sdlg} & .  & . & . & T_{N} \end{bmatrix}
\end{equation}

in which $T$ and $F$ are size-transition (i.e. survival and growth) and reproduction kernels drawn from our vital rate estimates for each species and endophyte status.


\subsubsection*{Statistical analysis of vital rates}
We modeled the effect of endophyte symbiosis on the mean and variance of vital rates by fitting generalized linear mixed models (GLMM) to the long-term data with year and plot random effects. We fit all vital rate models in a hierarchical Bayesian framework using Rstan \citep{Stan2022}, allowing us to propagate uncertainty from the vital rate estimates to our population model \citep{elderd2016quantifying}. 

The probability of survival and flowering are recorded as successes or failures and consequently are modeled as Bernoulli processes. We modeled growth (measured as the number of tillers in year t+1), and the number of flowering tillers with the zero-truncated Poisson-Inverse Gaussian distribution, and the number of spikelets per inflorescence with the Negative Binomial distribution. Each of these size-dependent vital rates are modeled with the same structure of linear predictor ($\mu$)

For example, growth ($G_{i,t1})$ of a given individual (i) in year t+1 is modeled as:
\begin{equation} 
\label{eq:growth}
\begin{aligned}
G_{i,t1} \sim P(IG(\mu_{s,e},\lambda_{s,e} )) \\
\end{aligned}
\end{equation}

Similarly, survival {$S_{i,t1}$} in year t+1 is modeled as:
\begin{equation} 
\label{eq:survival}
\begin{aligned}
S_{i,t1} \sim Bernoulli(\mu_{s,e}) \\
\end{aligned}
\end{equation}

Where $\mu$, for each species (s), is a linear function of the logarithm of plant size in year t (t), the plot level endophyte status (e), whether the plant was part of the initial transplanting or naturally recruited into the plot (r), along with random effects to account for plot(p), and year random effects specific to each species and endophyte status. Thus $\mu$ can be written:
\begin{equation}
\label{eq:linearpredictor}
\begin{aligned}
\mu_{s,e} = \beta^1_{s} + \beta^2_{s}log(size_{t}) + \beta^3_{s,e} + \beta^4_{r} \\
+  \tau + \rho \\
\tau \sim N(0,\sigma_{s,e}) \\
\rho \sim N(0,\sigma_{p})
\end{aligned}
\end{equation}

For all species, we account for a reproductive delay by modeling seedling growth and survival separately from adult growth and survival. Seedlings are those plants that are recruited into the plot in a given year, and typically have only one tiller. So, for seedlings, growth ($G^{sdlg}_{1,t1}$) is modelled as:

\begin{equation} 
\label{eq:sdlg_growth}
\begin{aligned}
G^{sdlg}_{1,t1} \sim P(IG(\mu^{sdlg}_{s,e},\lambda_{s,e} )) \\
\end{aligned}
\end{equation}

Similarly, survival ($S^{sdlg}_{i,t1}$) in year t+1 is modeled as:
\begin{equation} 
\label{eq:sdlg_survival}
\begin{aligned}
S^{sdlg}_{1,t1} \sim Bernoulli(\mu^{sdlg}_{s,e}) \\
\end{aligned}
\end{equation}

Here, $\mu^{sdlg}_{s,e}$ is the linear function for these seedling specific vital rates. It does not include size-dependence or an effect to account for the initial benefits of greenhouse rearing. 

\begin{equation}
\label{eq:sdlg_linearpredictor}
\begin{aligned}
\mu^{sdlg}_{s,e} = \beta^1_{s} + \beta^3_{s,e} \\
+ \tau + \rho \\
\tau \sim N(0,\sigma_{s,e}) \\
\rho \sim N(0,\sigma_{p})
\end{aligned}
\end{equation}

We ran each vital rate model for 2500 warm-up and 2500 MCMC sampling iterations with three chains using rStan. We assessed model convergence with trace plots of posterior chains and  and other model diagnostic statistics \citep{}. For those models that show poor convergence, we extended the MCMC sampling to include 5000 warm-up and 5000 sampling iterations, which was only necessary for seedling growth. For each of these vital rate models, we graphically check model fit with posterior predictive checks comparing simulated data from 500 posterior draws with the observed data (See supplement?). These checks provide evidence that our models are accurately recreating size-specific growth, survival, and reproduction patterns in our data across endophyte treatments.

We calculate the effect of endophytes on mean and variance in population growth rates by assembling matrix models with and without endophyte symbiosis following equation [\ref{eq:matrixmodel}]. To do this, we sample 500 posterior draws from our vital rate models, calculating yearly population growth rates with 50 samples from the joint posterior distribution of the year random effects. We then calculate the mean and variance of these annual growth rates for symbiotic and non-symbiotic populations

\subsubsection*{Stochastic growth rate simulation experiment}
We decompose the contribution of endophyte symbiosis to long-term stochastic growth rates by randomly sampling 500 annual population growth rates and calculating the geometric mean for with models including endophyte effects on mean, on variance, and on mean and variance compared to those without endophyte effects {\color{red}\citep{tuljapurkar1980population}.can add better citations for this}

\begin{equation}
\label{eq:stoch_sim}
\begin{aligned}
log(\lambda_s) ~ E[log(\Sigma(\mathbf{n}_{t+1})/\Sigma(\mathbf{n}_{t}))]
\end{aligned}
\end{equation}

\subsection*{Estimating climate drivers of environmental context-dependence}
\subsubsection*{Climate data}
\subsubsection*{Climate-explicit Model description and estimation}
\subsubsection*{Climate-explicit Model assessment}
\subsubsection*{Forecasting under alternative climate forcings}


We used statistics

}


\showmatmethods{} % Display the Materials and Methods section

\acknow{Please include your acknowledgments here, set in a single paragraph. Please do not include any acknowledgments in the Supporting Information, or anywhere else in the manuscript.}

\showacknow{} % Display the acknowledgments section

% Bibliography
\bibliography{endo_stoch_demo.bib}

\end{document}
