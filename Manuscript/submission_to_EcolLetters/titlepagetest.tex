%Version 2.1 April 2023
% See section 11 of the User Manual for version history
%
%%%%%%%%%%%%%%%%%%%%%%%%%%%%%%%%%%%%%%%%%%%%%%%%%%%%%%%%%%%%%%%%%%%%%%
%%                                                                 %%
%% Please do not use \input{...} to include other tex files.       %%
%% Submit your LaTeX manuscript as one .tex document.              %%
%%                                                                 %%
%% All additional figures and files should be attached             %%
%% separately and not embedded in the \TeX\ document itself.       %%
%%                                                                 %%
%%%%%%%%%%%%%%%%%%%%%%%%%%%%%%%%%%%%%%%%%%%%%%%%%%%%%%%%%%%%%%%%%%%%%

%%\documentclass[referee,sn-basic]{sn-jnl}% referee option is meant for double line spacing

%%=======================================================%%
%% to print line numbers in the margin use lineno option %%
%%=======================================================%%

\documentclass[lineno,Numbered]{sn-jnl}% Basic Springer Nature Reference Style/Chemistry Reference Style

%%======================================================%%
%% to compile with pdflatex/xelatex use pdflatex option %%
%%======================================================%%

%%\documentclass[pdflatex,sn-basic]{sn-jnl}% Basic Springer Nature Reference Style/Chemistry Reference Style


%%Note: the following reference styles support Namedate and Numbered referencing. By default the style follows the most common style. To switch between the options you can add or remove “Numbered” in the optional parenthesis. 
%%The option is available for: sn-basic.bst, sn-vancouver.bst, sn-chicago.bst, sn-mathphys.bst. %  
%% \documentclass[sn-nature]{sn-jnl}% Style for submissions to Nature Portfolio journals
%%\documentclass[sn-basic]{sn-jnl}% Basic Springer Nature Reference Style/Chemistry Reference Style
%%\documentclass[sn-mathphys,Numbered]{sn-jnl}% Math and Physical Sciences Reference Style
%%\documentclass[sn-aps]{sn-jnl}% American Physical Society (APS) Reference Style
%%\documentclass[sn-vancouver,Numbered]{sn-jnl}% Vancouver Reference Style
%%\documentclass[sn-apa]{sn-jnl}% APA Reference Style 
%%\documentclass[sn-chicago]{sn-jnl}% Chicago-based Humanities Reference Style
%%\documentclass[default]{sn-jnl}% Default
%%\documentclass[default,iicol]{sn-jnl}% Default with double column layout

%%%% Standard Packages
%%<additional latex packages if required can be included here>

\usepackage{graphicx}%
\usepackage{multirow}%
\usepackage{amsmath,amssymb,amsfonts}%
\usepackage{amsthm}%
\usepackage{mathrsfs}%
\usepackage[title]{appendix}%
\usepackage{xcolor}%
\usepackage{textcomp}%
\usepackage{manyfoot}%
\usepackage{booktabs}%
\usepackage{algorithm}%
\usepackage{algorithmicx}%
\usepackage{algpseudocode}%
\usepackage{listings}%
\usepackage{natbib}



%%%%

%%%%%=============================================================================%%%%
%%%%  Remarks: This template is provided to aid authors with the preparation
%%%%  of original research articles intended for submission to journals published 
%%%%  by Springer Nature. The guidance has been prepared in partnership with 
%%%%  production teams to conform to Springer Nature technical requirements. 
%%%%  Editorial and presentation requirements differ among journal portfolios and 
%%%%  research disciplines. You may find sections in this template are irrelevant 
%%%%  to your work and are empowered to omit any such section if allowed by the 
%%%%  journal you intend to submit to. The submission guidelines and policies 
%%%%  of the journal take precedence. A detailed User Manual is available in the 
%%%%  template package for technical guidance.
%%%%%=============================================================================%%%%

%\jyear{2021}%

\raggedbottom
%%\unnumbered% uncomment this for unnumbered level heads


%%=====================================================%%
%% redefining the title page to include author contributions
\makeatletter
\newcommand\@contrib{}%
\newcommand\@availab{}%


\def\statementsheadfont{\reset@font\fontsize{7bp}{9.5bp}\bfseries\selectfont\titraggedcenter}%
\def\statementssubheadfont{\reset@font\fontsize{7bp}{9.5bp}\bfseries\selectfont}%
\def\statementsfont{\reset@font\fontsize{7bp}{9.5bp}\selectfont\leftskip=24pt\rightskip=24pt\parfillskip=0pt plus 1fil}%




\newcommand\contribhead{\@startsection {section}{1}{\z@}{-22pt \@plus0ex \@minus0ex}{3pt}{\statementsheadfont}}%
\newcommand\contribsubhead{\@startsection{subsection}{2}{\z@}{3pt \@plus0ex \@minus0ex}{-.5em}{\statementssubheadfont}}

\newcommand\availabhead{\@startsection {section}{1}{\z@}{-22pt \@plus0ex \@minus0ex}{3pt}{\statementsheadfont}}%
\newcommand\availabsubhead{\@startsection{subsection}{2}{\z@}{3pt \@plus0ex \@minus0ex}{-.5em}{\statementssubheadfont}}


\newcommand\contribname{Author Contributions}%

\newcommand\availabname{Data and Code Accessibility}%


\long\def\contrib#1{\def\@contrib{%
		\let\paragraph\contribsubhead%
		\statementsfont%
		\contribhead*{\contribname}%
		#1\par}}%
	
\def\printcontrib{\ifx\@contrib\empty\else\@contrib\fi\par}%


\long\def\availab#1{\def\@availab{%
		\let\paragraph\availabsubhead%
		\statementsfont%
		\availabhead*{\availabname}%
		#1\par}}%

\def\printavailab{\ifx\@availab\empty\else\@availab\fi\par}%

\def\printarttype{\ifx\@arttype\empty\else\@arttype\fi\par}%

%
%%Article type
\def\arttypename{Article Type}%
\def\arttype#1{\ifx#1\empty\else\def\@arttype{\par\addvspace{10pt}{\keywordfont{\bfseries\arttypename:} #1\par}}\fi}%
\def\@arttype{}%


\def\printrunning{\ifx\@running\empty\else\@running\fi\par}%

%
%% Running title
\def\runningname{Running Title}%
\def\running#1{\ifx#1\empty\else\def\@running{\par\addvspace{10pt}{\keywordfont{\bfseries\runningname:} #1\par}}\fi}%
\def\@running{}%


\def\printfilecounts{\ifx\@filecounts\empty\else\@filecounts\fi\par}%

%
%% File Counts
\def\filecountsname{This file contains}%
\def\filecounts#1{\ifx#1\empty\else\def\@filecounts{\par\addvspace{10pt}{\keywordfont{\bfseries\filecountsname:} #1\par}}\fi}%
\def\@filecounts{}%


\renewcommand{\@maketitle}{\newpage\null%
	\if@remarkboxon\vbox to 0pt{\vspace*{-78pt}\hspace*{-10pt}\FMremark}\else\vskip21pt\fi%%\par%
	\hsize\textwidth\parindent0pt%%%\vskip7pt%
	%% Aritle Type
	{\hbox to \textwidth{{\Artcatfont\ArtType\hfill}\par}}
	%% Aritle Title
	\ifx\@title\empty\else%
	\removelastskip\vskip20pt\nointerlineskip%
	{\Titlefont\@title\par}
	%\addcontentsline{toc}{chapter}{\@title}% for bookmarks
	\fi%
	%% Aritle SubTitle
	\ifx\@subtitle\empty\else%
	\vskip9pt%
	{{\SubTitlefont\@subtitle\par}}
	\fi%
	%% Aritle Authors, Address and Correspondings
	\ifnum\aucount>0
	\global\punctcount\aucount%
	\vskip20pt%
	\artauthors\par%%     authors and emails
	{\vskip7pt\addressfont\auaddress\par%%      corresponding adress
		\removelastskip\vskip24pt%
		\ifnum\emailcnt>0\relax%
		\ifx\corrauthemail\@empty\else{\ifnum\aucount>1*\fi}%
		Corresponding author(s). E-mail(s): \corrauthemail\ \par\fi%
		\ifx\authemail\@empty\else Contributing authors:\ \authemail\fi%
		\\Phone: 719-359-2960
		\fi%
		\ifequalcont{\par$^{\dagger}$\@equalconttext\par}\fi%
		\removelastskip\vskip24pt%
		\ifpresentaddress{\par\@presentaddresstext\par}\fi%
	}
	\fi%
	\vspace{-20pt}
	{\printcontrib\par}%
	\vspace{-10pt}
	{\printavailab\par}%
	\vspace{40pt}
	{\printarttype\par}%
	{\printrunning\par}%
	{\printkeywords\par}%
	{\printfilecounts\par}%
	\newpage
	{\printabstract\par}%
	\newpage
	\ifx\@pacs\empty\else%
	\loop\ifnum\PacsCount>0%
	\csname\romannumeral\PacsTmpCnt StorePacsTxt\endcsname\par%
	\StepDownCounter{\PacsCount}%
	\StepUpCounter{\PacsTmpCnt}%
	\repeat%
	\fi%
	%%{\printhistory\par}%
	%%{\ifx\@motto\empty\else\@motto\fi}%
	\removelastskip\vskip36pt\vskip0pt}%

\makeatother






\begin{document}
	
	\title[Microbial symbionts buffer hosts from the demographic costs of environmental stochasticity]{Microbial symbionts buffer hosts from the demographic costs of environmental stochasticity}
	
	\running{Symbiont-mediated demographic buffering}
	\arttype{Letter}
	%%=============================================================%%
	%% Prefix	-> \pfx{Dr}
	%% GivenName	-> \fnm{Joergen W.}
	%% Particle	-> \spfx{van der} -> surname prefix
	%% FamilyName	-> \sur{Ploeg}
	%% Suffix	-> \sfx{IV}
	%% NatureName	-> \tanm{Poet Laureate} -> Title after name
	%% Degrees	-> \dgr{MSc, PhD}
	%% \author*[1,2]{\pfx{Dr} \fnm{Joergen W.} \spfx{van der} \sur{Ploeg} \sfx{IV} \tanm{Poet Laureate} 
		%%                 \dgr{MSc, PhD}}\email{iauthor@gmail.com}
	%%=============================================================%%
	
	\author*[1,2]{\fnm{Joshua C.} \sur{Fowler}}\email{jcf221@miami.edu}
	
	\author[3]{\fnm{Shaun} \sur{Ziegler}}\email{shaun.ziegler@gmail.com}
	
	
	\author[3]{\fnm{Kenneth D.} \sur{Whitney}}\email{whitneyk@unm.edu}
	
	\author[3]{\fnm{Jennifer A.} \sur{Rudgers}}\email{jrudgers@unm.edu}
	
	\author[1]{\fnm{Tom E.X.} \sur{Miller}}\email{tom.miller@rice.edu}
	
	
	
	\affil*[1]{\orgdiv{Department of BioSciences}, \orgname{Rice University}, \orgaddress{\city{Houston}, \postcode{77005}, \state{TX}, \country{USA}}}
	
	
	\affil[2]{\orgdiv{Department of Biology}, \orgname{University of Miami}, \orgaddress{ \city{Miami}, \postcode{33146}, \state{FL}, \country{USA}}}
	
	\affil[3]{\orgdiv{Department of Biology}, \orgname{University of New Mexico}, \orgaddress{\city{Albuquerque}, \postcode{87131}, \state{NM}, \country{USA}}}
	
	%%==================================%%
	%% Author contributions  %%
	%%==================================%%
	\contrib{J.C.F. contributed to data collection, data analysis, and led manuscript drafting.
		S.Z. contributed to data collection and manuscript revisions.
		K.D.W. contributed to research conception, data collection, and manuscript revisions.
		J.A.R. established transplant plots, contributed to research conception, data collection, and manuscript revisions.
		T.E.X.M. contributed to research conception, data collection, data analysis, and manuscript revisions.}
		

	%%==================================%%
	%% Data accessibility  %%
	%%==================================%%
	\availab{Data will be made accessible as an Environmental Data Initiative package  online \textbf{DOI: updated here when available.}
	Code for all analysis is available through \textbf{add github repo}}
	
	%%==================================%%
	%% sample for unstructured abstract %%
	%%==================================%%
	
	\abstract{	Species' persistence in increasingly variable future climates will depend on resilience against environmental stochasticity, which typically reduces fitness.
		Most organisms host microbiota that shield against stressful conditions, but it remains unknown whether microbial symbioses buffer hosts against the fitness costs of stochasticity because experiments must span long-term environmental variability. 
		We conducted a 14-year symbiont-removal experiment with seven host species to parameterize stochastic demographic models that predict both the mean and variance of host fitness. 
		We used cool season grasses and \emph{Epichlo\"{e}} fungal endophytes as a model system.
		Symbiotic fungal endophytes reduced variance in the fitness of grass hosts by $>$ 10\% on average across species, with up to 50\% reductions in fitness variance for some hosts.
		Hosts with ``fast'' life history traits that lacked longevity as an intrinsic buffer experienced the greatest benefits from symbiont-mediated buffering. 
		The contributions of variance buffering to host-symbiont mutualism were modest under the current climate regime compared to symbiont benefits to mean fitness. 
		However, simulations of increased environmental stochasticity amplified the benefits of variance buffering, which surpassed the symbionts' mean effects, which have dominated most prior research.
		These results establish microbial-mediated variance buffering as an important, yet cryptic, mechanism of resilience to increasing stochasticity under global change.
	}
	
	%%================================%%
	%% Sample for structured abstract %%
	%%================================%%
	
	% \abstract{\textbf{Purpose:} The abstract serves both as a general introduction to the topic and as a brief, non-technical summary of the main results and their implications. The abstract must not include subheadings (unless expressly permitted in the journal's Instructions to Authors), equations or citations. As a guide the abstract should not exceed 200 words. Most journals do not set a hard limit however authors are advised to check the author instructions for the journal they are submitting to.
		% 
		% \textbf{Methods:} The abstract serves both as a general introduction to the topic and as a brief, non-technical summary of the main results and their implications. The abstract must not include subheadings (unless expressly permitted in the journal's Instructions to Authors), equations or citations. As a guide the abstract should not exceed 200 words. Most journals do not set a hard limit however authors are advised to check the author instructions for the journal they are submitting to.
		% 
		% \textbf{Results:} The abstract serves both as a general introduction to the topic and as a brief, non-technical summary of the main results and their implications. The abstract must not include subheadings (unless expressly permitted in the journal's Instructions to Authors), equations or citations. As a guide the abstract should not exceed 200 words. Most journals do not set a hard limit however authors are advised to check the author instructions for the journal they are submitting to.
		% 
		% \textbf{Conclusion:} The abstract serves both as a general introduction to the topic and as a brief, non-technical summary of the main results and their implications. The abstract must not include subheadings (unless expressly permitted in the journal's Instructions to Authors), equations or citations. As a guide the abstract should not exceed 200 words. Most journals do not set a hard limit however authors are advised to check the author instructions for the journal they are submitting to.}
	
	\keywords{stochasticity, demography, symbiosis, mutualism, Epichlo\"{e}}
	
    \filecounts{Abstract ( XX words), Main Text (XX words), Figures (1-4); Supporting information Methods, Supporting Figures A1-A28, Supplemental Tables S1-S4}
	
	%%\pacs[JEL Classification]{D8, H51}
	
	%%\pacs[MSC Classification]{35A01, 65L10, 65L12, 65L20, 65L70}
	
	\maketitle
	
	
	
	
	
	\section*{Introduction}
	Global climate change involves increases in environmental variability, including changes to precipitation patterns and the frequency of extreme weather events \cite{seneviratne2012changes, ipcc_2021}.
	Yet, the ecological consequences of increased variability are less well understood than those of changing climate means, such as long-term warming or drying. 
	Incorporating environmental variability into forecasts of population dynamics can improve predictions of the future.
	
	Classic theory predicts that long-term population growth rates (equivalently, population mean fitness) will decline under increased environmental stochasticity because the costs of bad years outweigh the benefits of good years -- a consequence of nonlinear averaging \cite{lewontin_population_1969,tuljapurkar_population_1982}.
	For example, in unstructured populations, the long-term stochastic growth rate in a fluctuating environment ($\lambda_s$) will always be lower than the average growth rate ($\overline{\lambda}$) by an amount proportional to the environmental variance ($\sigma^2$): 
	\begin{equation}
		log(\lambda_s)  \approx log(\overline{\lambda}) - \frac{\sigma^2}{2\overline{\lambda}^2}
	\end{equation}
	
	\noindent Populations structured by size or stage similarly experience costs of variability \cite{cohen1979comparative, tuljapurkar2013population}.
	There are accordingly two pathways to increase population viability in a variable environment: increase the mean growth rate and/or dampen temporal fluctuation in growth rates, also called ``variance buffering''.
	
	Both the characteristics of species and the properties of their environment can buffer demographic fluctuations, including life history traits such as longevity \cite{pfister1998patterns, morris2008longevity}, correlations among vital rates \cite{compagnoni2016effect}, transient shifts in population structure \cite{ellis2013role}, the magnitude of environmental variability \cite{rodriguez2021limits}, or the degree of environmental autocorrelation \cite{tuljapurkar1980population,fieberg2001stochastic}. 
	These factors determine the risks of extinction faced by populations \cite{menges2000applications} and underlie management strategies promoting ecosystem resilience \cite{kuparinen2016fishing}. 
	Yet little is known about how biotic interactions influence demographic
	
	
	\newpage
	
	\bibliography{endo_stoch_demo}% common bib file
	%% if required, the content of .bbl file can be included here once bbl is generated
	%%\input sn-article.bbl
	
	
	\end{document}
	